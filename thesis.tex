%% Select the dissertation mode on
% See the documentation for more information about the available class options
% If you give option 'draft' or 'draft*', the draft mode is set on
\documentclass[dissertation,draft]{aaltoseries}
\usepackage[utf8]{inputenc}
% Lipsum package generates bullshit
\usepackage{lipsum}
% Set the document languages
\usepackage[finnish,swedish,english]{babel}

% The author of the dissertation
\author{Jaakko Luttinen}
% The title of the thesis
\title{Latent Gaussian models for Bayesian spatio-temporal modelling}

\begin{document}

%% The abstract of the dissertation in English
% Use this command!
\draftabstract{\lipsum[1-3]}
% Let's add another one in Finnish
\draftabstract[finnish]{\lipsum[4-6]}
% And yet another one in Swedish
\draftabstract[swedish]{\lipsum[7-9]}

%% Preface
% If you write this somewhere else than in Helsinki, use the optional location.
\begin{preface}[Optional location (if not defined, Helsinki)]
\lipsum[1-4]
\end{preface}

%% Table of contents of the dissertation
\tableofcontents

%% For article dissertations, remove if you write a monograph dissertation.
\listofpublications

%% Add lists of figures and tables as you usually.

%% Add list of abbreviations, list of symbols, etc., using your preferred
%% package/method.


%% The main matter, one can obviously use \input or \include


* Introduction
** Problem
** Outline
** Contributions

* Bayesian inference
** Probability theory?
** Variational Bayes
*** Variational message passing
** Markov chain Monte carlo
** Conclusions

* Spatio-temporal modelling
** Factor analysis
** State-space models
** Gaussian processes
** Conclusions

* Advanced models
** Outliers
** Global structure
** Local structure
** Time-varying dynamics
** Conclusions

* Efficient inference
** Parameter expansion
** Modular implementation
** Conclusions

* Discussion

\chapter{Introduction}

\section{Problem}
efficient, scale, robust

\section{Outline}

\section{Contributions}



\chapter{Bayesian inference}

\section{Probability theory}

\section{Variational Bayes}
\subsection{Variational message passing}

\section{Markov chain Monte Carlo}
hmc

\section{Conclusions}



\chapter{Spatio-temporal modelling}

\section{Factor analysis}

\section{State-space models}
linear and non-linear

\section{Gaussian processes}
sparse covfunc, inducing inputs

\section{Conclusions}



\chapter{Advanced models}

\section{Outliers}
robust pca

\section{Global structure}
GPFA

\section{Local structure}
aistats

\section{Time-varying dynamics}
spde

\section{Conclusions}




\chapter{Efficient inference}

\section{Parameter expansion}

\section{Modular implementation}
bayespy

\section{Conclusions}



\chapter{Discussion}
Jou! \cite{Luttinen:2013}





%% Examples of article references, remove these from your manuscript!
% Uncomment them, if you want to see the results of these commands in this example document

 % Refer to the Journal paper 1 of this example document
%\citepub{j1} \& \cpub{j1} \& \cp{j1} \& \pageref{j1} \& \ref{j1}

% Refer to the Conference paper of this example document
%\citepub[p.~2]{c1} \& \cpub[Sec.~ 1]{c1} \&  \cp[pp.~1--2]{c1} \& \pageref{c1} \& \ref{c1} 




\bibliographystyle{plain}
\bibliography{bibliography/bibliography.bib}


%% The following commands are for article dissertations, remove them if you write a monograph dissertation.

% Errata list, if you have errors in the publications.
\errata

%% The first publication (journal article)
% Set the publication information.
% This command musts to be the first!
\addpublication
{Journal Paper Authors}
{Journal Paper Title}
{Journal Name}
{Volume, issue, pages, and other detailed information}
{Month}
{Year}
{Copyright Holder}
{j1}
% Add the dissertation author's contribution to that publication (the order can be interchanged with \adderrata).
\addcontribution{ The author did this and that }
% Add the errata of the publication, remove if there are none (the order can be interchanged with \addauthorscontribution).
\adderrata{This is wrong}
% Add the publication pdf file, the filename is the parameter (must be the last).
\addpublicationpdf{articles/bayespy.pdf}

%% The second publication (conference article, note the optional parameter)
% Set the publication information.
\addpublication
[conference]
{Conference Paper Authors}
{Conference Paper Title}
{Conference Name}
{Location, pages, and other detailed information}
{Month}
{Year}
{Copyright Holder}
{c1}
% Add the dissertation author's contribution to that publication.
\addcontribution{The author did also this and that}
% No errata
% Add the publication pdf file, the filename is the parameter.
\addpublicationpdf{articles/bayespy.pdf}

%% The third publication (another journal paper, accepted for publication, note the optional parameter)
% Set the publication information, detailed information can be empty
\addpublication
[accepted]
{Journal Paper 2 Authors}
{Journal Paper 2 Title}
{Journal Name 2}
{}
{Month}
{Year}
{Copyright Holder}
{j2}
% Add the dissertation author's contribution to that publication.
\addcontribution{The author did everything}
% Add the errata of the publication, remove if there are none.
\adderrata{This is wrong}
% Add the publication pdf file, the filename is the parameter.
\addpublicationpdf{articles/bayespy.pdf}

%% The fourth publication (yet another journal paper, submitted for publication, note the optional parameter)
%% Note that you are allowed to use this option only when submitting the dissertation for pre-examination!
% Set the publication information, detailed information is not printed
\addpublication
[submitted]
{Journal Paper 3 Authors}
{Journal Paper 3 Title}
{Journal Name 3}
{}
{Submission date}
{Year}
{No copyright holder at this moment}
{j3}
% Add the dissertation author's contribution to that publication.
\addcontribution{The author did everything}
% Add the errata of the publication, remove if there are none. (in submitted paper this is unlikely)
%\adderrata{This is wrong}
% Add the publication pdf file, the filename is the parameter.
\addpublicationpdf{articles/bayespy.pdf}


\end{document}


%%% Local Variables: 
%%% TeX-command-default: "Make"
%%% mode: latex
%%% TeX-master: "thesis"
%%% End: 
